%--------------------------------------------------------------------
\section{Conclusion}

We have given an overview of Fresnel, a browser-independent, extensible vocabulary for modeling Semantic Web presentation knowledge. Fresnel has been designed as a modularized, declarative language manipulating selection, formatting, and styling concepts that are applicable across representation paradigms and output formats. We have presented applications implementing Fresnel core modules while based on different representation and navigation paradigms, thus substantiating the claim that Fresnel can be used to model presentation knowledge that is reusable across browsers and other Semantic Web visualization tools.

Although core modules have been frozen for the time being, the Fresnel vocabulary remains a work in progress as new extension modules meeting special needs are being developed  (e.g., for describing the {\em purpose} of lenses and for {\em editing} information). Extension modules are not necessarily aimed at being application- and paradigm-independent, as they might not be relevant in all cases; but their inclusion in Fresnel provides users with a unified framework for modeling presentation knowledge. Another field for future work is enabling Fresnel formats and lenses to be retrieved transparently from the Web so that RDF browsers could query the Web for display knowledge about previously unknown vocabularies.

The development of Fresnel is an open, community-based effort and new contributors are welcome to participate in it. More information can be found on its Web site \rdf{http://www.w3.org/2005/04/fresnel-info/}. 


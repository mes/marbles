% Fresnel paper for ISWC05

% based on LLNCS.DEM the demonstration file of
% the LaTeX macro package from Springer-Verlag
% for Lecture Notes in Computer Science,
% version 2.2 for LaTeX2e
%
\documentclass{llncs}
%
\usepackage{makeidx}  % allows for indexgeneration
\usepackage{graphicx}
%
\begin{document}
%
\newcommand{\rdf}[1]{{\small \texttt{#1}}}

\frontmatter          % for the preliminaries
%
\pagestyle{headings}  % switches on printing of running heads
\mainmatter              % start of the contributions
%
\title{Fresnel - A Browser-Independent Display Vocabulary for RDF}
%
\titlerunning{Fresnel - A browser-independent Display Vocabulary for RDF}
% abbreviated title (for running head)
%                                     also used for the TOC unless
%                                     \toctitle is used
%
\author{Christian Bizer\inst{1} \and Ryan Lee\inst{2} \and Stefano Mazzocchi\inst{3} \and Emmanuel Pietriga\inst{4}}
%
%\authorrunning{Bizer et al.}   % abbreviated author list (for running head)
%
%%%% modified list of authors for the TOC (add the affiliations)
\tocauthor{Chris Bizer (Berlin),
Ryan Lee (MIT),
Stefano Mazzocchi (MIT Libraries),
Emmanuel Pietriga (INRIA)}
%
\institute{Freie Universit\"at Berlin, Germany \\
\email{chris@bizer.de}
\and
W3C/MIT CSAIL, Cambridge, USA\\
\email{ryanlee@w3.org}
\and
MIT Libraries, Cambridge, USA\\
\email{stefanom@mit.edu}
\and
INRIA \& Laboratoire de Recherche en Informatique (LRI), Orsay, France\\
\email{emmanuel.pietriga@inria.fr}
}

\maketitle

%--------------------------------------------------------------------
\begin{abstract}
Todo: Change for new structure of paper: 1. There are different approaches, 2. Abstract Selection and Styling as common problems, 3. Developed Fresnel as a browser-independent ontology to handle these problems in order to faciliate the exchange of display knowledge.

Presenting Semantic Web content in a human-readable way consists in addressing two issues: specifying {\em what} information contained in an RDF graph should be presented and {\em how} this information should be presented. Each RDF browser or visualization tool currently relies on its own ad hoc mechanisms and vocabularies for addressing these issues, making it impossible to share RDF presentation knowledge across applications. Recognizing the general need for displaying RDF content and wanting to avoid reinventing the wheel with each new tool, we developed Fresnel as a browser-independent vocabulary of core RDF display concepts applicable across different representation paradigms. Fresnel's two foundational concepts are lenses and styles. Lenses define which properties of an RDF resource, or group of related resources, are displayed and how those properties are ordered. Styles determine how resources and properties are rendered by specifying styling attributes, making use of existing languages such as CSS and SVG wherever possible. In this paper describe the Fresnel display vocabulary and show how Fresnel is used within different RDF browsers, including Longwell and IsaViz.
\end{abstract}

%--------------------------------------------------------------------
%--------------------------------------------------------------------
\section{Introduction}

Software agents are the primary consumers of Semantic Web content. RDF is thus designed to facilitate machine interpretability of information and does not define a visual presentation model since human readability is not one of its stated goals. Todo: Put this in: 1) observe that RDF applications do not always need to do lots of semantic processing (as in the typical RDF-is-for-machines-view), but often only need to show the information in the RDF repository in a human friendly way....However, content encoded in RDF has to be viewed and understood by humans on many occasions. Displaying RDF in a human-friendly manner is therefore a legitimate concern, addressed by various types of applications using different representation paradigms. Tools like IsaViz \cite{isaviz}and Welkin \cite{Welkin} represent RDF models as node-link diagrams, where the subjects and objects of RDF triples are the nodes, and predicates the arcs, of the graph. Other tools use nested box layouts (Longwell \cite{simile}) or table-like layouts (Brownsauce \cite{Steer03}, Noadster \cite{Rutledge05}, Swoop \cite{MindSwap05}) for displaying properties of RDF resources with varying levels of details. A third approach combines these paradigms and extends them with specialized user interface widgets designed for specific information items like DNA sequences, calendar data or tree structures (Haystack \cite{Quan04}, mSpace\cite{mspace2005}).


\subsection{Selection and Styling}

Seen from an abstract perspective  generating a visualization is a 5 step process:
1. Select which resources to display.
2. Select which properties to display and how to display property values (DBview/Slice).
3  Select a presentation paradigm
4. Style what has been selected (like CSS).
5. Generate output in the appropriate format.

Step 1 depends on the browsing-paradigm and is not generalizable.
Step 3 and 5 depends on the browser and target device and is not generalizable.
Step 2 and 4 are generalizable. Thats why we generalize them in order to faciliate the exchange of knowledge.

% Approaches to the display of RDF data are based on different representation paradigms, but all of them consider the whole operation as a two-step process: content selection and content styling. Todo: Don't talk about process.

Todo: Use the text below to make the points above clear.

Providing a single and global view of an RDF model is often not useful. The amount of data makes it difficult to extract information relevant to the current task and is a cognitive overload. The first step thus consists in restricting the visualization to small but cohesive parts of the RDF graph. Users can then select other points of interest by navigating in the model through hyperlinks and refine the selection with paradigms such as faceted browsing \cite{simile}. Todo refer to DB view or MMS slice\cite{Isakowitz:1995:RMS} as similar concepts.

Identifying what content to show is not sufficient to get a human-friendly presentation of the information. To achieve this goal, the selected content items have to be laid out properly and rendered with graphical attributes that favor legibility in order to facilitate general understanding of the displayed information. Relying solely on the content's structure and exploiting knowledge contained in the schema associated with the data is not sufficient to produce sophisticated visualizations. The second step thus consists in styling selected content items using explicit styling instructions found in stylesheets.

Relying solely on the content's structure and exploiting knowledge contained in the schema associated with the data is not sufficient to produce sophisticated visualizations. ... That why a way to encode additional display knowledge is needed ...

\subsection{Outline of the paper}

In section 2 we discuss the various approaches proposed so far for representing display knowledge neccessary for rendering visualizations of RDF graphs and retrieve requirements from them(good and bad paractices).  Then we abstract two common problems that are handled in an ad hoc manner by all approaches: Selection and Styling. Based on this abstraction we have developed fresnel as a browser-indpendent...exchange of knowledge.... The Fresnel lens and style vocabularies are described in sections 2 and 3. Section 4 describes the selector languages (having a growing expressiviness) which can be used within Fresnel to identify the elements of RDF graphs lenses and styles should apply to. Section 5 shows how Fresnel can be used within different browsers, including Longwell and IsaViz.


%--------------------------------------------------------------------
%--------------------------------------------------------------------
\section{Related Work}

Early RDF visualization tools rendered RDF models in a predefined, non-customizable way \cite{Steer03}. Recent tools provide more flexible visualizations, which can be customized by writing style sheets, transformations or templates for specific RDF vocabularies, following either a declarative or a procedural approach.

Procedural approaches encode presentation knowledge as a series of transformation steps. One approach from this category is using XSLT to transform RDF/XML-encoded RDF graphs in an environment such as Cocoon \cite{cocoon05}. Authoring XSLT templates and XPath expressions to handle arbitrary RDF/XML is complex, if not impossible, considering the many potential serializations of a given RDF graph and the present lack of a commonly accepted RDF canonicalization in XML \cite{Carroll04}. This problem is addressed by Xenon \cite{quan05}, an RDF stylesheet ontology that builds on the ideas of XSLT, but combines recursive templating mechanisms with SPARQL as an RDF-specific selector language. Xenon succeeds in addressing XSLT's RDF canonicalization problem but still has the drawback - as all procedural approaches - that transformation rules are very closely bound to a specific display paradigm or output format preventing the reuse of presentation knowledge across applications. 

Declarative approaches represent presentation knowledge as a set of generic selection and formatting instructions; trying to copy the ideas of HTML and CSS which where successful for the classic Web. One example from this category is the Haystack Slide ontology \cite{HaystackUI03}, which is used to describe how Haystack display widgets are laid out. A further example are IsaViz's node-and-arc oriented Graph Style Sheets\cite{gss03}. All declarative approaches are having in common that they encode presentation knowledge in a way which is closely bound to the display paradigm and presentation capabilities of the browser for which they have been developed and thus is not very meaningful for other browsers.

%--------------------------------------------------------------------
%--------------------------------------------------------------------
\section{Fresnel Lens Vocabulary}

Fresnel lenses describe which properties of RDF resources are shown and how these properties are ordered. For example, a summary lens for a class representing people such as \rdf{foaf:Person} might display the name and the email address properties of each person. 

The \rdf{fresnel:lensDomain} property specifies the set of instances to which a lens is applicable. A lens domain can be defined by one or more classes, in which case the lens is applicable to instances of these classes. A domain can also be defined by a set of instances using an FSL or SPARQL selector (see section \ref{selectors}). Such selectors are used to specify a lens domain in terms of the existence (or lack) of properties and values associated with the resources that constitute the domain. They thus make it possible to associate lenses with untyped RDF resources, which can and do occur in real-world models as \rdf{rdf:type} properties are not mandatory. 

In a distributed and open environment, browsers confronted with unknown RDF vocabularies for which no stylesheet is available locally can query repositories for appropriate lenses. Queries are made on the \rdf{fresnel:lensDomain} property of available lenses in order to evaluate their ability to handle the resources to be presented. Queries can also make use of the \rdf{fresnel:purpose} property, which encodes metadata about lenses, more specifically their intended use and characteristics. The purpose property can state that a lens is the default lens for a given class, or that it gives a good one-line summary (e.g. a label) of resources, etc. The following example shows a lens applicable to resources identifying persons, as defined by the Friend-of-a-Friend (FOAF) vocabulary \cite{foaf}.
% try to make sure this example stays on one page
\begin{small}
\begin{verbatim}
:PersonLens a fresnel:Lens ;
    fresnel:lensDomain foaf:Person ;
    fresnel:purpose fresnel:defaultLens ;
    fresnel:showProperties ( 
        foaf:name
        foaf:mbox
        [ a fresnel:PropertyDescription ;
          fresnel:property foaf:knows ;
          fresnel:depth "2"^^xsd:nonNegativeInteger ;
          fresnel:sublens :PersonLens ]
        fresnel:allProperties ) ;
    fresnel:hideProperties ( 
        rdfs:label 
        dc:title ) .
\end{verbatim}
\end{small}


\subsection{Property Selection and Ordering}

In the previous example\footnote{All examples in this paper use the Notation 3 syntax for RDF \cite{N3}.}, \rdf{fresnel:showProperties} and \rdf{fresnel:hideProperties} define which properties must be shown or hidden when the lens is used to display resources of type \rdf{foaf:Person}. The value of \rdf{showProperty} and \rdf{hideProperties} can either be a single URI reference identifying the property to show or hide, or a list of such URI references. Properties to show are displayed according to their order of appearance in the list that is the value of \rdf{fresnel:showProperties}.

Special value \rdf{fresnel:allProperties} can be used to avoid having to explicitly name each property that should be displayed. This value is also useful when the list of properties that can potentially be associated with resources handled by a lens is unknown to the lens' author but should nevertheless be displayed. When it appears as a member of the list of properties to be shown by a lens, \rdf{fresnel:allProperties} designates the set of properties that are not explicitly designated by other property URI references in the list, except for properties that appear in the list of properties to hide (\rdf{fresnel:hideProperties}). These unnamed properties are displayed according to the position of \rdf{fresnel:allProperties} in the list. In the previous example, \rdf{foaf:name}, \rdf{foaf:mbox} and \rdf{foaf:knows} properties are displayed in this order, before all other properties, which appear next as indicated by the presence of \rdf{fresnel:allProperties} at the end of the list of properties to be shown. Properties \rdf{rdfs:label} and \rdf{dc:title} will not be displayed even if they exist, as they are explicitly declared as hidden.

Fresnel provides two additional constructs for specifying what properties of resources to display. The first one handles the potential irregularity of RDF data coming from the fact that different authors might use similar terms coming from different vocabularies to make equivalent statements. Sets of such similar properties can be grouped in ordered lists and said to be \rdf{fresnel:alternateProperties}. For instance, \rdf{foaf:name}, \rdf{dc:title}, \rdf{rdfs:label} can be considered by a lens as giving the same information about resources. A browser using this lens will then try to display the resource's \rdf{foaf:name}. If the latter does not exist, the browser will look for \rdf{dc:title} and \rdf{rdfs:label} in this order. The second Fresnel construct, \rdf{fresnel:mergeProperties}, is used to merge the values of related properties (e.g. \rdf{foaf:homepage} and \rdf{foaf:workHomepage}) into one single set of values for presentation purposes.

\subsection{Lenses and Sublenses}

It is often desirable to display cohesive parts of RDF graphs involving a group of related resources, such as a person together with her projects and papers. Cohesive parts of graphs are defined by relating lenses using the \rdf{fresnel:sublens} property. In the previous example, a sublens is used to display persons that are known by the current person. The \rdf{fresnel:sublens} property states that lens \rdf{:PersonLens} should be used to display the value of \rdf{foaf:knows} properties. As this introduces recursiveness in the lens definition, property \rdf{fresnel:depth} is used to specify the closure value. In the end, this example lens will display persons together with her friends and the persons known by her friends.

As shown in this example, lens URIs are used to explicitly identify what lens to use as sublenses for presenting property values. Appropriate sublenses can also be identified by expressing constraints on lens definitions (domain, purpose, etc.) using a Fresnel Selector Expression or a SPARQL query evaluated against the set of lenses available to the browser.

%Naming the properties to show ultimately results in an output of statements with a lens-matched resource as the subject of each statement.  The objects of the output statements may also be resources; to describe how to display statement objects, a {\em sublens} may be used.


%--------------------------------------------------------------------
\section{Fresnel Style Vocabulary}
\label{fsStyle}

Fresnel lenses specify what information to display but do not give any indication about how to display it. The representation of selected information items mainly depends on the browser's representation paradigm (e.g. nested box layout, table layout, node-link diagrams, advanced widget-based UI, etc.) which defines a default rendering method. The final rendering can however be customized by associating specific styling and layout instructions to elements of the representation, as CSS styling rules do for elements of HTML documents.

Fresnel styling rules are made of a selector and a set of associated styling instructions. Following the lens approach, property \rdf{fresnel:styleDomain} takes a selector as its value (see section \ref{selectors}) and is used to declare the set of RDF properties or resources to which styling instructions apply. Styling instructions themselves are expressed using either Fresnel constructs for RDF-specific styling concepts, or existing languages (CSS \cite{CSS} and SVG \cite{SVG}) for fonts, colors, margins, and borders. 

\subsection{Fresnel Styling Instructions}

Fresnel styling instructions are expressed in RDF and can be used to specify RDF-specific styling concepts such as how properties are labeled and grouped. Fresnel's default behavior is to label properties using the declared \rdf{rdfs:label} of the property type. If this information is not available, then the property's URI is displayed instead. This behavior can be customized thanks to property \rdf{fresnel:label}, which defines whether a property label is displayed (\rdf{fresnel:show}) or not (\rdf{fresnel:none}, see figure \ref{styleCode}), or if it has a fixed custom literal value.

The presentation of property values can also be customized. Fresnel's default property value display method consists in searching for a lens matching the value's type and characterized by purpose \rdf{fresnel:labelLens}. Some kinds of values are however better represented by other means than labels, and it is therefore desirable to specify alternate behaviors. Instruction \rdf{fresnel:value} can be associated with different Fresnel behaviors, such as \rdf{fresnel:image} (see figure \ref{styleCode}) which states that the value to be represented is a URI reference identifying a bitmap image  whose content should be fetched over the Web in order to be displayed. Property values can be grouped, and additional content such as commas and an ending period can be specified using instructions \rdf{fresnel:contentAfter} and \rdf{fresnel:contentLast} to present multi-valued properties. Instruction \rdf{fresnel: contentNoValue} can be used to generate fixed content signaling missing values.

\subsection{CSS and SVG Styling Instructions}

As the set of display elements and layout method depend entirely on the browser's representation paradigm, Fresnel only defines an abstract representation model that can be interpreted and instantiated differently by every application. This abstract model is used to identify the elements of the output document to which CSS and SVG styling instructions can be hooked in a cross-application and cross-format manner. In the abstract model, the container box is the outermost Fresnel container. It identifies the region of the display that contains a group of principal resources to be presented. The container box contains resource and property boxes which identify regions of the display that are respectively associated with resources themselves and their properties. Property boxes are further decomposed as a label box and one or more value boxes which respectively hold the property's label and its value(s). Styling instructions can then be hooked to these presentation elements. The instructions are specified as literal values of properties \rdf{fresnel:containerStyle}, \rdf{resourceStyle}, \rdf{propertyStyle}, \rdf{labelStyle} and \rdf{valueStyle}. Such literal values can contain either styling instructions directly (see figure \ref{styleCode}), or a CSS class name referencing a rule in an external stylesheet.

\begin{figure}
    \begin{center}
      \includegraphics[width=10cm]{boxmodelexample.pdf}
    \end{center}
    \vspace{-0.3cm}
    \caption{Box model with attached styling instructions}
    \label{boxModel}
\end{figure}

Figure \ref{boxModel} contains a schematic view of how the abstract Fresnel model could be interpreted and instantiated by a Web-based browser using the nested box representation paradigm. 

\begin{figure}
    \begin{tabular}{c|c}
      \includegraphics[height=4.3cm]{boxmodelexampleoutput.pdf} &
      \includegraphics[height=4.3cm]{isv_scs.pdf} \\
      (a) & (b)\\
    \end{tabular}
    \caption{Two interpretations for two representation paradigms}
    \label{boxEx}
\end{figure}

Figure \ref{boxEx}-a shows a possible final rendering based on this model as it might be generated by Longwell. Note that the example model of figure \ref{boxModel} is just one possible instantiation of the abstract Fresnel model. The abstract presentation elements introduced above could be mapped to (possibly non-rectangular) shapes laid out in very different manners, depending on the browser's capabilities and fundamental representation paradigm.

Figure \ref{boxEx}-b shows a different final rendering produced by IsaViz\footnote{This Fresnel presentation was simulated with a GSS stylesheet \cite{Pietriga03} as IsaViz does not support all Fresnel features at the time of submission.}, based on another instantiation of the Fresnel abstract representation model relying on node-link diagrams. The two examples of figure \ref{boxEx} thus illustrate two presentations, by two browsers based on different representation paradigms, of the same data using the same Fresnel lens and styling instructions. But other representation models based on  other paradigms are possible. For instance, a server producing HTML pages for user agents that do not support CSS might choose a basic table-based representation of RDF triples and interpret the property and value boxes as cells of a table row.


The style example of figure \ref{styleCode} specifies that \rdf{foaf:depiction} property values should be displayed as images with a black border, but no property label. Additional styling instructions are used to further customize \rdf{foaf:depiction} properties' appearance. Some of these instructions might be ignored by browsers depending on their relevance with respect to the underlying representation para\-digm. For instance, the border-top instruction will be interpreted by browsers based on a standard box model, whereas it is likely to be ignored by visualization tools such as IsaViz which is more likely to interpret stroke-related instructions.

\vspace{-0.5cm}

\begin{figure}
\begin{small}
\begin{verbatim}
:depictStyle rdf:type fresnel:Style ;
    fresnel:styleDomain foaf:depiction ;
    fresnel:label fresnel:none ;
    fresnel:value fresnel:image ;
    fresnel:propertyStyle "border-top: solid 
                           black"^^fresnel:StylingInstructions;
    fresnel:labelStyle "stroke-dasharray: 10px,20px; stroke-width: 2px; 
                        stroke: red"^^fresnel:StylingInstructions ;
    fresnel:valueStyle "border: solid black"^^fresnel:StylingInstructions.
\end{verbatim}
\end{small}
\vspace{-0.3cm}
\caption{Fresnel style for property \rdf{foaf:depiction}}
\label{styleCode}
\end{figure}



%--------------------------------------------------------------------
\section{Fresnel Selectors}
\label{selectors}

Selection in Fresnel occurs at different levels: when specifying what properties should be displayed or hidden (\rdf{fresnel:showProperties}, \rdf{fresnel:hidePrope\-rties}), and when specifying the domain of a lens or style (\rdf{fresnel:lensDomain}, \rdf{fresnel:styleDomain}). All four properties take as values expressions that identify elements of the RDF model to be presented, in other words specific nodes and arcs in the graph. Three languages can be used in Fresnel for specifying selection expressions, offering increasing levels of expressive power and complexity. 

\subsection{Basic Selectors}

The simplest selectors in Fresnel take the form of URI references. Depending on its context of use, a URI reference can identify a resource, a class of resources, or a type of property, as shown in the following examples.
\begin{small}
\begin{verbatim}:PersonLens a fresnel:Lens ;
    fresnel:lensDomain foaf:Person ;
    fresnel:showProperties ( foaf:name
                             foaf:depiction ).
\end{verbatim}
\end{small}
Property \rdf{fresnel:lensDomain} indicates that this lens should be used to display instances of class \rdf{foaf:Person}, i.e., resources that have an \rdf{rdf:type} property pointing at class (or at a subclass of \footnote{Subclass and subproperty relationships must be taken into account by the selection mechanism, provided that an RDF Schema or OWL ontology is available at runtime for the associated vocabulary.}) \rdf{foaf:Person}. Property \rdf{fresnel:show\-Properties} takes as its value a list of URIs referencing RDF property types. In this example, properties \rdf{foaf:name} and \rdf{foaf:depiction} of resources displayed by this lens should be shown.

Basic selectors simply name the type of resources or properties that should be selected. They are easy to use but have a very limited expressive power. For instance, they cannot be used to specify that a lens should apply to all instances of class foaf:Person that are the subject of at least five \rdf{foaf:knows} statements (i.e., all resources representing persons that know more than five other persons). Such selectors, and more complex ones, can be expressed with the languages described in the next two sections.

\subsection{Fresnel Selector Language}

The Fresnel Selector Language (FSL) is a language for modeling traversal paths in RDF graphs, designed to address the specific requirements of a selector language for Fresnel. It does not pretend to be a full so-called RDFPath language, but tries to be as simple as possible. Still trying to avoid reinventing the wheel, FSL is strongly inspired by XPath, reusing many of its concepts and syntactic constructs while adapting them to RDF's graph-based data model. RDF models are considered as directed labeled graphs according to RDF Concepts and Abstract Syntax \cite{rdfcas04}. FSL is therefore fully independent from any serialization.

An FSL expression represents a path from a node or arc to another node or arc, passing by an arbitrary number of other nodes and arcs. FSL paths explicitly represent both nodes and arcs as steps on the path, as it is desirable to be able to constrain the type of arcs a path should traverse (something that is not relevant in XPath as the only relation between the nodes of an XML tree is the parent-child relation which bears no explicit semantics).

Each step on the path, called a location step, follows the XPath location step syntax and is made of a) an optional axis declaration specifying the traversal direction in the directed graph, b) a type test taking the form of a URI reference represented as an XML qualified name (QName), or a \rdf{*} when the type is left unconstrained, c) optional predicates that specify further conditions on the nodes and arcs to be matched by this step.

The type test constrains property arcs to be labeled with the URI represented by the QName, or resource nodes to be instances of the class identified by this QName. In other words, type tests specify constraints on the types of properties and classes of resources to be traversed and selected by paths. Constraints on the URI of resources can be expressed as predicates associated with node location steps. A consequence of interpreting QName tests as type constraints is that FSL is syntactically and semantically compatible with Fresnel's basic selectors. The latter can therefore be considered a very limited subset of what can be expressed with FSL. Thus, any valid basic selector expression is a valid FSL expression.

More information about the language, including its grammar, data model and semantics can be found on the Fresnel Web site \cite{fsl05}. In the following examples, literals containing FSL expressions should all declare \rdf{fresnel:selector} as their datatype. This declaration has been omitted here for clarity.

\begin{small}
\begin{verbatim}
# A lens for foaf:Person resources that know at least five other resources
:PersonLens a fresnel:Lens ;
    fresnel:lensDomain "foaf:Person[count(foaf:knows) >= 5]".

# Show the foaf:name property of all foaf:Person
# instances known by the current resource.
:PersonLens a fresnel:Lens ;
    fresnel:showProperties ("foaf:knows/foaf:Person/foaf:name").
\end{verbatim}
\end{small}

\subsection{SPARQL}

The SPARQL RDF query language \cite{sparql05} offers the highest expressive power and can be used to specify lens and style domains. SPARQL queries used in this context must always return exactly one node set, meaning that only one variable is allowed in the query's SELECT clause. As with previous FSL examples, the literal's datatype declaration has been omitted.

\begin{small}
\begin{verbatim}
# A lens for John Doe's mailboxes
:PersonLens a fresnel:Lens ;
    fresnel:lensDomain "SELECT ?mbox WHERE ( ?x foaf:name 'John Doe' )
                                           ( ?x foaf:mbox ?mbox )".
\end{verbatim}
\end{small}




%--------------------------------------------------------------------
%--------------------------------------------------------------------
\section{Implementations}
\label{impl}

Fresnel has been designed as an application- and output format-independent RDF presentation vocabulary. In this section we give an overview of various applications implementing Fresnel: Longwell \cite{simile} and Horus \cite{horus} which both render RDF data as HTML Web pages using nested box layouts, IsaViz \cite{isaviz} which represents RDF graphs as node-link diagrams, and Cardovan, a browser and lens editor based on the SWT GUI toolkit.

\vspace{1em}
{\bf Longwell} is a Web-based RDF browser whose foundational navigation paradigm is faceted browsing. Faceted browsing displays only the properties that are configured to be 'facets' (i.e., to be important for the user browsing data in one or more specific domains) using values for those fields as a means for zooming into a collection by selecting those items with a particular field-value pair.

\begin{figure}
\begin{center}
\includegraphics[width=12cm]{longwellScreens.png}
\end{center}
\vspace{-2em}
\caption{Displaying a view of an organization (left) and a constituent member (right) in Longwell}
\label{longwellFig1}
\vspace{-1em}
\end{figure}

The latest version of Longwell relies on the SIMILE Fresnel rendering engine, a Java library built on the Sesame triple store. The engine implements all of the Fresnel core vocabulary and the portion of the extended vocabulary relating to linking groups to CSS stylesheets as well as the option of using FSL as a selector language.
% The emphasis of the Fresnel engine is on building a correct implementation. 
The Fresnel engine output consists solely of an XML representation of the Fresnel lenses and formats as they apply to one resource. Longwell then applies an XSLT transformation to the XML to generate XHTML.  The default XSLT stylesheet shipped with Longwell will generate a traditional nested box layout, as Horus does, but the stylesheet can be modified by XSLT developers to change the model as they see fit.

The left side of Figure \ref{longwellFig1} shows the rendering of a \rdf{foaf:Organization} resource using a lens that gives some details about the organization and lists its constituent members, all \rdf{foaf:Person}s, each listed with their corresponding nickname information to assist in identification.

The nickname list for each person is preceded by the string 'aka: ', added to the display by using the \rdf{fresnel:contentFirst} directive.  The list is also comma separated, accomplished by setting \rdf{fresnel:contentAfter} to a comma.  Clicking on a URI in the display brings the user to that URI; clicking on a textual label changes Longwell's focus to the resource represented by that label.

On the right side of Figure \ref{longwellFig1}, the focus is on one specific member of the organization featured in the left side.  A sublens is used to generate office contact details, and the same sublens used in the organization focus (left image) to describe an organization's members is used in the person focus (right image) to describe who this person claims to know.


\vspace{1em}
{\bf Horus} is an RDF browser that displays RDF information using a nested box layout. The browser provides a simple navigation paradigm for selecting RDF resources and allows users to switch between different lenses for rendering the resources. Horus supports Fresnel lenses and formats, which can be associated together using Fresnel groups. Groups can refer to external CSS style sheets which are used to define fonts, colors and borders. Horus supports basic selectors, but does not offer SPARQL and FSL as selector languages. Horus is implemented using PHP and is backed by a MySQL database. Applying a lens to an RDF resource results in an intermediate tree, which is formatted afterwards using the formats that are associated to the group of the selected lens. The ordered and formatted intermediate tree is then serialized into XHTML.  

\begin{figure}
\begin{tabular}{cc}
\includegraphics[width=6cm]{horus1.pdf} \hspace{0.02cm} &
\includegraphics[width=6cm]{horus2.pdf} \\
\end{tabular}
\vspace{-1em}
\caption{Two different views on the same person in Horus: detailed view (left), friends view (right)}
\label{horusFig1}
\vspace{-1.1em}
\end{figure}

Figure \ref{horusFig1} shows two different views on the same person in Horus. The view on the left uses a lens that displays many details about persons. The sentence "{\em This person knows the following people}" is a custom label for property \rdf{foaf:knows}. The disclaimer "{\em That a person knows somebody does\ldots}" is static content added using property \rdf{fresnel:contentLast}. Some of the links are formatted as external links (\rdf{fresnel:value} formatting instruction set to \rdf{fresnel:externalLink}), while others refer to RDF resources in the knowledge base, and thus have a different rendering.

%\begin{figure}
%\begin{center}
%\begin{tabular}{c}
%\includegraphics[width=10cm]{horus1.pdf} \\
%\includegraphics[width=10cm]{horus2.pdf}
%\end{tabular}
%\end{center}
%\caption{Two different views on the same person in Horus: detailed view (left), friends view (right)}
%\label{horusFig1}
%\end{figure}


On the right side of Figure \ref{horusFig1}, the same person is shown using a different lens. This lens displays less details about the person itself, but refers to a second lens (used as a sublens) for displaying details about other persons known by this person. As the sublens belongs to a different group, another CSS class is used to style the names of the person's friends.


\vspace{1em}
{\bf IsaViz} is an RDF authoring environment representing RDF models as node-link diagrams. The interpretation of Fresnel in IsaViz is inspired by both Generalized Fisheye Views \cite{furnas06} and Magic Lenses \cite{bier93}. Fresnel lenses, in conjunction with the formats associated with them through groups, are considered as ``genuine'' lenses that modify the visual appearance of objects below them.

Figure \ref{isvFresnelFig} (left) shows the default rendering of a region of an RDF model containing a \rdf{foaf:Person} resource. At this level of magnification, only a few of the many property values associated with the resource are visible. Users need to navigate in the graph in order to get to the values of properties, which can be cumbersome. Alternatively, users can select a Fresnel lens from the list of available lenses loaded in IsaViz through the graphical user interface. The selected lens is then tied to the mouse cursor, and when the lens hovers over a resource that matches its domain, the resource's visual appearance gets modified according to the lens and associated format(s). Resources that match the selected lens' domain are made visually prominent by rendering all other nodes and all arcs using shades of gray with minimum contrast. When the lens hovers over a resource, properties selected by the lens are temporarily rendered with highly-contrasted vivid colors and brought within the current view,
closer to the main resource and reordered clockwise according to the ordering of properties in the lens definition, as illustrated in Figure \ref{isvFresnelFig} (right). Property values revert back to their original state when the lens moves away from the resource. All these visual modifications, including color and position changes, are smoothly animated thanks to the underlying graphical toolkit's animation capabilities \cite{pietriga05}, thus keeping the user's cognitive load low following the principles of perceptual continuity.

\begin{figure}
\begin{tabular}{cc}
\includegraphics[height=4.45cm]{isavizscreen1.png} \hspace{0.04cm} &
\includegraphics[height=4.45cm]{isavizscreen2.png} \\
\end{tabular}
\vspace{-1em}
\caption{Zoomed-in view of a \rdf{foaf:Person} resource in IsaViz: default presentation (left) and rendered with a Fresnel lens (right)}
\label{isvFresnelFig}
%\vspace{-1em}
\end{figure}

Fresnel core formatting instructions are interpreted as customizations of the original layout and rendering of nodes and links in the diagram. For instance, nodes representing \rdf{foaf:image} property values can be rendered by fetching the actual image from the Web, as illustrated in Figure \ref{isvFresnelFig} (right). The default labels of nodes and arcs can be customized using \rdf{fresnel:label} instructions. In case a resource is the subject of multiple statements involving the same property or properties defined as \rdf{fresnel:mergeProperties}, the arcs and nodes representing these statements can be merged as a single arc and node with all values within that node, optionally separated by text as specified in \rdf{fresnel:contentBefore}, \rdf{fresnel:contentAfter} and related formatting instructions.


\vspace{1em}
{\bf Cardovan} is IBM's implementation of Fresnel lenses (see Figure \ref{cardovanFig}). Written in Java, Cardovan renders lenses with the SWT graphical user interface toolkit.  Cardovan is similar to other implementations in that it uses a subset of CSS to specify the layout of lens components on the screen. A remarkable feature of Cardovan is that it allows users to modify a lens in place. Users can add new properties to the lens, modify property values, and rearrange the physical layout of the properties displayed, though it is not a full WYSIWYG Fresnel lens designer. The project is still in its early stages, but is functional and is already being used for internal projects at IBM.

\begin{figure}
\begin{center}
\includegraphics[width=12.2cm]{cardovan.png}
\end{center}
\vspace{-2em}
\caption{Editing a lens (left) and visualizing the result (right) in Cardovan}
\label{cardovanFig}
\end{figure}


%--------------------------------------------------------------------
%--------------------------------------------------------------------
% $Id: fresnelInLongwell.tex 1370 2005-04-20 20:34:09Z ryanlee $
\subsection{Longwell}

Longwell is a suite of RDF browsing applications written in Java and developed by the SIMILE project \cite{simile}.  Initial releases of Longwell implemented an {\em ad hoc} display vocabulary, amended as application needs arose.  The next major stable release, currently under development, will instead support the core selection terms from Fresnel.  Graph results from searches will be filtered through Fresnel lens definitions before being passed on to the rendering engine.  Also, a basic, experimental RDF browser, @@@unnamed, based off the same set of selection and browsing code, will implement all of the core Fresnel terms.

(@@@assuming a screenshot of longwell2 shows off the same data used in the IsaViz screenshot) As seen in the screenshot of Longwell \ref{longwellScreen}, the display of a \rdf{foaf:Person} using the (@@@which lens?) renders only what is specified in the lens in the order of specification; there is more data in the underlying collection involving contact information that is not shown.

Longwell generates XHTML and uses CSS to style data according to the CSS box model; this portion does not use Fresnel styling.


%--------------------------------------------------------------------
%--------------------------------------------------------------------
\subsection{IsaViz}

IsaViz \cite{isaviz} already implements a rendering engine that can interpret GSS style\-sheets \cite{Pietriga03} for styling RDF models represented as node-link diagrams. The new version (currently under development) features two presentation paradigms that both support Fresnel stylesheets: the default paradigm based on node-link diagrams, and a new paradigm based on the box model described in section \ref{fsStyle} which will take advantage of the zoomable user interface and semantic zooming capabilities associated with the graphical toolkit upon which the user interface is built \cite{Pietriga05a}.

@@@ Outline of interpretation in the node-link diagram representation

@@@ Mention alternate interpretation in the Semantic ZUI representation


%--------------------------------------------------------------------
%--------------------------------------------------------------------
\subsection{Other Implementations}

Another effort to implement Fresnel is the Arago RDF-browser. Arago is currently being developed at DERI and renders XHTML representations of RDF data\cite{Gassert05}. Another groups that are currently in the process of evaluating Fresnel and migh support it in their tools are the Haystack team at MIT\cite{Quan04} and the Noadster team at CWI\cite{Rutledge05}.




%--------------------------------------------------------------------
%--------------------------------------------------------------------
\section{Conclusion}

We have given an overview of Fresnel, a browser-independent, extensible vocabulary for modeling Semantic Web content presentation knowledge. Fresnel has been designed as a modularized, declarative language manipulating selection, formatting, and styling concepts that are applicable across representation paradigms, layout methods, and output formats. Fresnel core modules can be used to model presentation knowledge that is compatible and reusable between browsers and other types of Semantic Web information visualization~tools.

%I've added a ~ between the last two words to prevent tools from going on the next line.

Although core modules have been frozen for the time being, the Fresnel vocabulary remains a work in progress as new extension modules meeting special needs are being developed (e.g., for describing the {\em purpose} of lenses or adding new formatting capabilities). Extension modules are not necessarily aimed at being application- and paradigm-independent, as they might not be relevant in all cases; but their inclusion in Fresnel provides users with a unified framework for modeling presentation knowledge. 

Core modules are currently being implemented in various types of applications: SIMILE's Longwell \cite{simile} faceted browser, IsaViz \cite{isaviz} which represents RDF graphs as node-link diagrams, and Horus \cite{horus}, a PHP-based RDF browser. More information about Fresnel can be found on its web site\footnote{http://www.w3.org/2005/04/fresnel-info/}. Its development is an open, community-based effort and new contributors are welcome to participate in it.




%
% ---- Bibliography ----
%

\bibliographystyle{splncs}
\bibliography{fresnel}

\end{document}

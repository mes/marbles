%--------------------------------------------------------------------
\section{Related Work}

Early RDF visualization tools rendered RDF models in a predefined, non-customizable way \cite{Steer03}. Recent tools provide more flexible visualizations, which can be customized by writing style sheets, transformations or templates for specific RDF vocabularies, following either a declarative or a procedural approach.

Procedural approaches encode presentation knowledge as a series of transformation steps. One approach from this category is using XSLT to transform RDF/XML-encoded RDF graphs in an environment such as Cocoon \cite{cocoon05}. Authoring XSLT templates and XPath expressions to handle arbitrary RDF/XML is complex, if not impossible, considering the many potential serializations of a given RDF graph and the present lack of a commonly accepted RDF canonicalization in XML \cite{Carroll04}. This problem is addressed by Xenon \cite{quan05}, an RDF stylesheet ontology that builds on the ideas of XSLT, but combines recursive templating mechanisms with SPARQL as an RDF-specific selector language. Xenon succeeds in addressing XSLT's RDF canonicalization problem but still has the drawback - as all procedural approaches - that transformation rules are very closely bound to a specific display paradigm or output format preventing the reuse of presentation knowledge across applications. 

Declarative approaches represent presentation knowledge as a set of generic selection and formatting instructions; trying to copy the ideas of HTML and CSS which where successful for the classic Web. One example from this category is the Haystack Slide ontology \cite{HaystackUI03}, which is used to describe how Haystack display widgets are laid out. A further example are IsaViz's node-and-arc oriented Graph Style Sheets\cite{gss03}. All declarative approaches are having in common that they encode presentation knowledge in a way which is closely bound to the display paradigm and presentation capabilities of the browser for which they have been developed and thus is not very meaningful for other browsers.
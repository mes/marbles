%--------------------------------------------------------------------
\section{Conclusion}

We have given an overview of Fresnel, a browser-independent, extensible vocabulary for modeling Semantic Web content presentation knowledge. Fresnel has been designed as a modularized, declarative language manipulating selection, formatting, and styling concepts that are applicable across representation paradigms, layout methods, and output formats. Fresnel core modules can be used to model presentation knowledge that is compatible and reusable between browsers and other types of Semantic Web information visualization~tools.

%I've added a ~ between the last two words to prevent tools from going on the next line.

Although core modules have been frozen for the time being, the Fresnel vocabulary remains a work in progress as new extension modules meeting special needs are being developed (e.g., for describing the {\em purpose} of lenses or adding new formatting capabilities). Extension modules are not necessarily aimed at being application- and paradigm-independent, as they might not be relevant in all cases; but their inclusion in Fresnel provides users with a unified framework for modeling presentation knowledge. 

Core modules are currently being implemented in various types of applications: SIMILE's Longwell \cite{simile} faceted browser, IsaViz \cite{isaviz} which represents RDF graphs as node-link diagrams, and Horus \cite{horus}, a PHP-based RDF browser. More information about Fresnel can be found on its web site\footnote{http://www.w3.org/2005/04/fresnel-info/}. Its development is an open, community-based effort and new contributors are welcome to participate in it.


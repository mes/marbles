% Fresnel paper for ISWC05

% based on LLNCS.DEM the demonstration file of
% the LaTeX macro package from Springer-Verlag
% for Lecture Notes in Computer Science,
% version 2.2 for LaTeX2e
%
\documentclass{llncs}
%
\usepackage{makeidx}  % allows for indexgeneration
\usepackage{graphicx}
%
\begin{document}
%
\newcommand{\rdf}[1]{{\small \texttt{#1}}}

\frontmatter          % for the preliminaries
%
\pagestyle{headings}  % switches on printing of running heads
\mainmatter              % start of the contributions
%
\title{Fresnel - A Browser-Independent Presentation Vocabulary for RDF}
%
\titlerunning{Fresnel - A browser-independent Presentation Vocabulary for RDF}
% abbreviated title (for running head)
%                                     also used for the TOC unless
%                                     \toctitle is used
%
\author{Christian Bizer\inst{1} \and Ryan Lee\inst{2} \and Emmanuel Pietriga\inst{3}}
%
%\authorrunning{Bizer et al.}   % abbreviated author list (for running head)
%
%%%% modified list of authors for the TOC (add the affiliations)
\tocauthor{Chris Bizer (Berlin),
Ryan Lee (MIT),
Emmanuel Pietriga (INRIA)}
%
\institute{Freie Universit\"at Berlin, Germany \\
\email{chris@bizer.de}
\and
W3C/MIT CSAIL, Cambridge, USA\\
\email{ryanlee@w3.org}
\and
INRIA \& Laboratoire de Recherche en Informatique (LRI), Orsay, France\\
\email{emmanuel.pietriga@inria.fr}
}

\maketitle

%--------------------------------------------------------------------
\begin{abstract}
Semantic Web browsers and other tools aimed at displaying RDF data to end users are all concerned with the same problem: presenting content primarily intended for machine consumption in a human-readable way. Their solutions differ but in the end address the same two high-level issues, no matter the underlying representation paradigm: specifying (i) {\em what} information contained in RDF models should be presented (content selection) and (ii) {\em how} this information should be presented (content formatting and styling). However, each tool currently relies on its own {\em ad hoc} mechanisms and vocabulary for specifying RDF presentation knowledge, making it difficult to share and reuse such knowledge across applications. Recognizing the general need for presenting RDF content to users and wanting to promote the exchange of presentation knowledge, we developed Fresnel as a browser-independent extensible vocabulary of core RDF display concepts.
\end{abstract}

%--------------------------------------------------------------------
%--------------------------------------------------------------------
\section{Introduction}

Software agents are the primary consumers of Semantic Web content. RDF is thus designed to facilitate machine interpretability of information and does not define a visual presentation model since human readability is not one of its stated goals. Todo: Put this in: 1) observe that RDF applications do not always need to do lots of semantic processing (as in the typical RDF-is-for-machines-view), but often only need to show the information in the RDF repository in a human friendly way....However, content encoded in RDF has to be viewed and understood by humans on many occasions. Displaying RDF in a human-friendly manner is therefore a legitimate concern, addressed by various types of applications using different representation paradigms. Tools like IsaViz \cite{isaviz}and Welkin \cite{Welkin} represent RDF models as node-link diagrams, where the subjects and objects of RDF triples are the nodes, and predicates the arcs, of the graph. Other tools use nested box layouts (Longwell \cite{simile}) or table-like layouts (Brownsauce \cite{Steer03}, Noadster \cite{Rutledge05}, Swoop \cite{MindSwap05}) for displaying properties of RDF resources with varying levels of details. A third approach combines these paradigms and extends them with specialized user interface widgets designed for specific information items like DNA sequences, calendar data or tree structures (Haystack \cite{Quan04}, mSpace\cite{mspace2005}).


\subsection{Selection and Styling}

Seen from an abstract perspective  generating a visualization is a 5 step process:
1. Select which resources to display.
2. Select which properties to display and how to display property values (DBview/Slice).
3  Select a presentation paradigm
4. Style what has been selected (like CSS).
5. Generate output in the appropriate format.

Step 1 depends on the browsing-paradigm and is not generalizable.
Step 3 and 5 depends on the browser and target device and is not generalizable.
Step 2 and 4 are generalizable. Thats why we generalize them in order to faciliate the exchange of knowledge.

% Approaches to the display of RDF data are based on different representation paradigms, but all of them consider the whole operation as a two-step process: content selection and content styling. Todo: Don't talk about process.

Todo: Use the text below to make the points above clear.

Providing a single and global view of an RDF model is often not useful. The amount of data makes it difficult to extract information relevant to the current task and is a cognitive overload. The first step thus consists in restricting the visualization to small but cohesive parts of the RDF graph. Users can then select other points of interest by navigating in the model through hyperlinks and refine the selection with paradigms such as faceted browsing \cite{simile}. Todo refer to DB view or MMS slice\cite{Isakowitz:1995:RMS} as similar concepts.

Identifying what content to show is not sufficient to get a human-friendly presentation of the information. To achieve this goal, the selected content items have to be laid out properly and rendered with graphical attributes that favor legibility in order to facilitate general understanding of the displayed information. Relying solely on the content's structure and exploiting knowledge contained in the schema associated with the data is not sufficient to produce sophisticated visualizations. The second step thus consists in styling selected content items using explicit styling instructions found in stylesheets.

Relying solely on the content's structure and exploiting knowledge contained in the schema associated with the data is not sufficient to produce sophisticated visualizations. ... That why a way to encode additional display knowledge is needed ...

\subsection{Outline of the paper}

In section 2 we discuss the various approaches proposed so far for representing display knowledge neccessary for rendering visualizations of RDF graphs and retrieve requirements from them(good and bad paractices).  Then we abstract two common problems that are handled in an ad hoc manner by all approaches: Selection and Styling. Based on this abstraction we have developed fresnel as a browser-indpendent...exchange of knowledge.... The Fresnel lens and style vocabularies are described in sections 2 and 3. Section 4 describes the selector languages (having a growing expressiviness) which can be used within Fresnel to identify the elements of RDF graphs lenses and styles should apply to. Section 5 shows how Fresnel can be used within different browsers, including Longwell and IsaViz.


%--------------------------------------------------------------------
%%--------------------------------------------------------------------
\section{Related Work}

Early RDF visualization tools rendered RDF models in a predefined, non-customizable way \cite{Steer03}. Recent tools provide more flexible visualizations, which can be customized by writing style sheets, transformations or templates for specific RDF vocabularies, following either a declarative or a procedural approach.

Procedural approaches encode presentation knowledge as a series of transformation steps. One approach from this category is using XSLT to transform RDF/XML-encoded RDF graphs in an environment such as Cocoon \cite{cocoon05}. Authoring XSLT templates and XPath expressions to handle arbitrary RDF/XML is complex, if not impossible, considering the many potential serializations of a given RDF graph and the present lack of a commonly accepted RDF canonicalization in XML \cite{Carroll04}. This problem is addressed by Xenon \cite{quan05}, an RDF stylesheet ontology that builds on the ideas of XSLT, but combines recursive templating mechanisms with SPARQL as an RDF-specific selector language. Xenon succeeds in addressing XSLT's RDF canonicalization problem but still has the drawback - as all procedural approaches - that transformation rules are very closely bound to a specific display paradigm or output format preventing the reuse of presentation knowledge across applications. 

Declarative approaches represent presentation knowledge as a set of generic selection and formatting instructions; trying to copy the ideas of HTML and CSS which where successful for the classic Web. One example from this category is the Haystack Slide ontology \cite{HaystackUI03}, which is used to describe how Haystack display widgets are laid out. A further example are IsaViz's node-and-arc oriented Graph Style Sheets\cite{gss03}. All declarative approaches are having in common that they encode presentation knowledge in a way which is closely bound to the display paradigm and presentation capabilities of the browser for which they have been developed and thus is not very meaningful for other browsers.

%--------------------------------------------------------------------
%--------------------------------------------------------------------
\section{Core Vocabulary Overview}
\label{fresnelov}

Fresnel is an RDF vocabulary, described by an OWL ontology \cite{fresnel05}. Fresnel presentation knowledge is thus expressed declaratively in RDF and relies on two foundational concepts: {\em lenses} and {\em formats} (see Figure \ref{foundationalConceptsFig}). Lenses specify which properties of RDF resources are shown and how these properties are ordered while formats indicate how to format content selected by lenses and optionally generate additional static content and hooks in the form of CSS class names that can be used to style the output through external CSS style sheets. The following sections introduce the main vocabulary elements using the examples in Figures \ref{exampleFig} and \ref{exampleFig2}.

\begin{figure}
\begin{center}
\includegraphics[width=10cm]{overview.pdf}
\caption{Fresnel foundational concepts}
\label{foundationalConceptsFig}
\end{center}
\end{figure}

Figure \ref{exampleFig} shows a simple lens and associated formats used to present information about a person described with the FOAF vocabulary \cite{foaf}. This figure also shows a possible rendering of such a resource, that a browser like Horus~\cite{horus} or Longwell~\cite{simile} could produce. Examples use the Notation 3 syntax \cite{N3}.

\begin{figure}
\begin{tabular}{lp{4.5cm}}
\begin{small}
\begin{tabular}{ll}
(300) & \rdf{ :PersonLens a fresnel:Lens ;} \\
(301) & \rdf{ ~fresnel:classLensDomain foaf:Person ;} \\
(302) & \rdf{ ~fresnel:showProperties (} \\
(303) & \rdf{ ~~foaf:name} \\
(304) & \rdf{ ~~foaf:mbox} \\
(305) & \rdf{ ~~[rdf:type fresnel:PropertyDescription;} \\
(306) & \rdf{ ~~~fresnel:alternateProperties (} \\
(307) & \rdf{ ~~~   foaf:depiction foaf:img p3p:image )} \\
(308) & \rdf{ ~~~] ) .} \\
 & \\
(309) & \rdf{ :nameFormat a fresnel:Format ; } \\
(310) & \rdf{ ~~~~fresnel:propertyFormatDomain foaf:name ;} \\
(311) & \rdf{ ~~~~fresnel:label "Name" .} \\
 & \\
(312) & \rdf{ :mboxFormat a fresnel:Format ;} \\
(313) & \rdf{ ~~~~fresnel:propertyFormatDomain foaf:mbox ;} \\
(314) & \rdf{ ~~~~fresnel:label "Mailbox" ;} \\
(315) & \rdf{ ~~~~fresnel:value fresnel:externalLink ;} \\
(316) & \rdf{ ~~~~fresnel:valueFormat [ fresnel:contentAfter "," ] .} \\
 & \\
(317) & \rdf{ :depictFormat a fresnel:Format ;} \\
(318) & \rdf{ ~~~~fresnel:propertyFormatDomain foaf:depiction ;} \\
(319) & \rdf{ ~~~~fresnel:label fresnel:none ;} \\
(320) & \rdf{ ~~~~fresnel:value fresnel:image .} \\
\end{tabular}
\end{small}
&
\begin{picture}(45,0.1)(0,0)\put(-26,67){\makebox(0,0){\includegraphics[width=4cm]{boxmodelexampleoutput.pdf}}}\end{picture} \\
\end{tabular}
\caption{A lens and some formats for presenting instances of class \rdf{foaf:Person}}
\label{exampleFig}
\vspace{-2em}
\end{figure}

\subsection{Content selection}

The domain of a lens indicates the set of resources to which a lens applies (line 301: the lens applies to instances of class \rdf{foaf:Person}). Property \rdf{fresnel:showProperties} is used to specify what properties of these resources to show and in what order (lines 302-308). In this example, the values of both \rdf{fresnel:classLensDomain} and \rdf{fres\-nel:showProperties} are basic selectors, which take the form of plain URIs (represented here as qualified names), respectively identifying the class of resources and property types to select. More advanced selection expressions can be written using either FSL or SPARQL. They make it possible to associate lenses with untyped RDF resources, which do occur in real-world models since \rdf{rdf:type} properties are not mandatory. They can also be used to specify that a lens should display all properties of a given namespace, or any other complex selection condition(s) that can be represented by an FSL or SPARQL expression (see Section \ref{selectors}).

Fresnel Core provides additional constructs for specifying what properties of resources to display. The special value \rdf{fresnel:allProperties} is used when the list of properties that can potentially be associated with resources handled by a lens is unknown to the lens' author but should nevertheless be displayed. When it appears as a member of the list of properties to be shown by a lens, \rdf{fresnel:allProperties} designates the set of properties that are not explicitly designated by other property URI references in the list, except for properties that appear in the list of properties to hide (\rdf{fresnel:hideProperties}). Two other constructs are used to handle the potential irregularity of RDF data stemming from the fact that different authors might use similar terms coming from different vocabularies to make equivalent statements. Sets of such similar properties can be said to be \rdf{fresnel:alternateProperties}. For instance, \rdf{foaf:depiction}, \rdf{foaf:img} and \rdf{p3p:image} could be considered as providing the same information about resources displayed by a given lens. A browser using this lens would try to display the resource's \rdf{foaf:depiction}. If the latter did not exist, the browser would then look for \rdf{foaf:img} or \rdf{p3p:image} (see lines 305-307). Such knowledge can also be represented through ontology mapping mechanisms, but Fresnel provides this alternative as the ontology layer should not be made a requirement of the Fresnel presentation process. The other construct, \rdf{fresnel:mergeProperties}, is used to merge the values of related properties (e.g. \rdf{foaf:homepage} and \rdf{foaf:work\-Homepage}) into one single set of values that can later be formatted as a whole. 

The presentation of property values is not limited to a single level, and (possibly recursive) calls to lenses can be made to display details about the value of a property. Lenses used in this context are referred to as {\em sublenses}. Modifying the example of Figure \ref{exampleFig}, we specify in Figure \ref{exampleFig2} that the browser should render values of the property \rdf{foaf:knows} (lines 405-407) using another lens (\rdf{PersonLabelLens}, lines 410-413). The FSL expression (see Section \ref{selectors}) on line 406 specifies in an XPath-like manner that only values of \rdf{foaf:knows} that are instances of \rdf{foaf:Person} should be selected.

The sublens mechanism implies that a lens can recursively call itself as a sublens for displaying property values. In order to prevent infinite loops caused by such recursive calls, Fresnel defines a closure mechanism that allows Fresnel presentation designers to specify the maximum depth of the recursion.

\subsection{Content formatting}

The default layout of selected information items is highly dependent on the browser's representation paradigm (e.g. nested box layout, node-link diagrams, etc.), but the final rendering can be customized by associating formatting and styling instructions with elements of the representation.

\begin{picture}(340,1)(0,-5)
  \put(-10,0){\line(1,0){340}}
\end{picture}
\begin{figure}
\vspace{-4em}
\begin{tabular}{lp{4.5cm}}
\begin{small}
\begin{tabular}{ll}
(400) & \rdf{:PersonLens a fresnel:Lens ;}\\
(401) & \rdf{ ~~fresnel:classLensDomain foaf:Person ;} \\
(402) & \rdf{ ~~fresnel:showProperties (} \\
(403) & \rdf{ ~~~~foaf:name} \\
(404) & \rdf{ ~~~~foaf:mbox} \\
(405) & \rdf{ ~~~~[rdf:type fresnel:PropertyDescription ;} \\
(406) & \rdf{ ~~~~~fresnel:property "foaf:knows[foaf:Person]"}\srdf{$^{\wedge\wedge}$fresnel:fslSelector;} \\
(407) & \rdf{ ~~~~~fresnel:sublens :PersonLabelLens] } \\
(408) & \rdf{ ~~) ;} \\
(409) & \rdf{ ~~fresnel:group :FOAFmainGroup .} \\
 & \\
(410) & \rdf{ :PersonLabelLens a fresnel:Lens ;} \\
(411) & \rdf{ ~~fresnel:classLensDomain foaf:Person ;} \\
(412) & \rdf{ ~~fresnel:showProperties ( foaf:name ) ;} \\
(413) & \rdf{ ~~fresnel:group :FOAFsubGroup .} \\
 & \\
(414) & \rdf{ :nameFormat a fresnel:Format ; } \\
(415) & \rdf{ ~~fresnel:propertyFormatDomain foaf:name ;} \\
(416) & \rdf{ ~~fresnel:label "Name" ;} \\
(417) & \rdf{ ~~fresnel:group :FOAFmainGroup .} \\
 & \\
(418) & \rdf{ :mboxFormat a fresnel:Format ;} \\
(419) & \rdf{ ~~fresnel:propertyFormatDomain foaf:mbox ;} \\
(420) & \rdf{ ~~fresnel:label "Mailbox" ;} \\
(421) & \rdf{ ~~fresnel:value fresnel:externalLink ;} \\
(422) & \rdf{ ~~fresnel:valueFormat [ fresnel:contentAfter "," ] ;} \\
(423) & \rdf{ ~~fresnel:group :FOAFmainGroup .} \\
 & \\
(424) & \rdf{ :friendsFormat a fresnel:Format ;} \\
(425) & \rdf{ ~~fresnel:propertyFormatDomain foaf:name ;} \\
(426) & \rdf{ ~~fresnel:label "Friends" ;} \\
(427) & \rdf{ ~~fresnel:group :FOAFsubGroup .} \\
 & \\
(428) & \rdf{ :FOAFmainGroup a fresnel:Group .} \\
(429) & \rdf{ :FOAFsubGroup a fresnel:Group .} \\
 \end{tabular}
\end{small}
&
\begin{picture}(45,10)(0,0)\put(-73,60){\makebox(0,0){\includegraphics[width=4.2cm]{boxmodelexampleoutput2.pdf}}}\end{picture} \\
\end{tabular}
\caption{An example of a lens using another lens to display some property values}
\label{exampleFig2}
\vspace{-2em}
\end{figure}

\begin{picture}(340,1)(0,-10)
  \put(-10,0){\line(1,0){340}}
\end{picture}

Formats apply to resources, or to properties and their values, depending on the specified domain. The three example formats of Figure \ref{exampleFig} apply respectively to the properties \rdf{foaf:name}, \rdf{foaf:mbox} and \rdf{foaf:depiction} (lines 310, 313, 318). Formats can be used to set properties' labels (lines 311, 314, 319). Property \rdf{fresnel:label} does not specify a particular layout but simply gives a text string that can be used to identify the property. Labels might already be defined for many properties (e.g., in the associated vocabulary description using \rdf{rdfs:label}), but such labels are not guaranteed to exist. Moreover, a given label might not always be the most appropriate depending on the context in which the property is displayed. For instance, the default label associated with property \rdf{foaf:name} in the FOAF schema is {\em name}. When displaying the persons known by the current person in Figure \ref{exampleFig2}, this default label is replaced by {\em Friends} (line 426) so as to indicate the appropriate interpretation of the corresponding \rdf{foaf:name} property values in this context. The customization of labels also proves useful when displaying property values that are not direct properties of the current resource, as is made possible by the use of SPARQL or FSL expressions such as:
\[ \rdf{foaf:knows/*[airport:iataCode/text() = 'CDG']/foaf:name} \]
which would require an explanatory label such as {\em Friends that leave near Paris}.

Formats can also give instructions regarding how to render values. For instance, line 315 indicates that \rdf{foaf:mbox} values should be rendered as clickable links (email addresses). Values of \rdf{foaf:depiction} should be fetched from the Web and rendered as bitmap images (line 320).

Property values can be grouped, and additional content such as commas and an ending period can be specified to present multi-valued properties (line 316: inserting a comma in-between each email address). CSS class names can also be associated with the various elements being formatted. These names appear in the output document and can be used to style the output by authoring and referencing CSS style sheets that use rules with the same class names as selectors.

\subsection{Lens and Format Grouping}

Lenses and formats can be associated through \rdf{fresnel:Group}s so that browsers can determine which lenses and formats work together.  Fresnel groups are taken into account by browsers when selecting what format(s) to apply to the data selected by a given lens, as several formats might be applicable to the same property values.

Figure \ref{exampleFig2} illustrates the use of Fresnel groups to display different labels for the \rdf{foaf:name} property depending on the context in which the property is shown: the property is labeled {\em Name} when displayed in the context of the \rdf{PersonLens} lens, but is labeled {\em Friends} when displayed in the context of the \rdf{PersonLabelLens} lens. This is achieved by associating the \rdf {PersonLens} (lines 400-409) and the \rdf{nameFormat} (lines 414-417) to one group: \rdf{FOAFmainGroup}, and by associating the \rdf{PersonLabelLens} (lines 410-413) and the \rdf{friendsFormat} (lines 424-427) to a second group: \rdf{FOAFsub\-Group}.

A Fresnel group can also serve as a placeholder for formatting instructions that apply to all formats associated with that group, thus making it possible to factorize the declarations. It is also typically used to declare group-wide data, relevant to both lenses and formats, such as namespace prefix bindings.


%--------------------------------------------------------------------
\section{Fresnel Selectors}
\label{selectors}

Selection in Fresnel occurs at different levels: when specifying what properties should be displayed or hidden (\rdf{fresnel:showProperties}, \rdf{fresnel:hidePrope\-rties}), and when specifying the domain of a lens or style (\rdf{fresnel:lensDomain}, \rdf{fresnel:styleDomain}). All four properties take as values expressions that identify elements of the RDF model to be presented, in other words specific nodes and arcs in the graph. Three languages can be used in Fresnel for specifying selection expressions, offering increasing levels of expressive power and complexity. 

\subsection{Basic Selectors}

The simplest selectors in Fresnel take the form of URI references. Depending on its context of use, a URI reference can identify a resource, a class of resources, or a type of property, as shown in the following examples.
\begin{small}
\begin{verbatim}:PersonLens a fresnel:Lens ;
    fresnel:lensDomain foaf:Person ;
    fresnel:showProperties ( foaf:name
                             foaf:depiction ).
\end{verbatim}
\end{small}
Property \rdf{fresnel:lensDomain} indicates that this lens should be used to display instances of class \rdf{foaf:Person}, i.e., resources that have an \rdf{rdf:type} property pointing at class (or at a subclass of \footnote{Subclass and subproperty relationships must be taken into account by the selection mechanism, provided that an RDF Schema or OWL ontology is available at runtime for the associated vocabulary.}) \rdf{foaf:Person}. Property \rdf{fresnel:show\-Properties} takes as its value a list of URIs referencing RDF property types. In this example, properties \rdf{foaf:name} and \rdf{foaf:depiction} of resources displayed by this lens should be shown.

Basic selectors simply name the type of resources or properties that should be selected. They are easy to use but have a very limited expressive power. For instance, they cannot be used to specify that a lens should apply to all instances of class foaf:Person that are the subject of at least five \rdf{foaf:knows} statements (i.e., all resources representing persons that know more than five other persons). Such selectors, and more complex ones, can be expressed with the languages described in the next two sections.

\subsection{Fresnel Selector Language}

The Fresnel Selector Language (FSL) is a language for modeling traversal paths in RDF graphs, designed to address the specific requirements of a selector language for Fresnel. It does not pretend to be a full so-called RDFPath language, but tries to be as simple as possible. Still trying to avoid reinventing the wheel, FSL is strongly inspired by XPath, reusing many of its concepts and syntactic constructs while adapting them to RDF's graph-based data model. RDF models are considered as directed labeled graphs according to RDF Concepts and Abstract Syntax \cite{rdfcas04}. FSL is therefore fully independent from any serialization.

An FSL expression represents a path from a node or arc to another node or arc, passing by an arbitrary number of other nodes and arcs. FSL paths explicitly represent both nodes and arcs as steps on the path, as it is desirable to be able to constrain the type of arcs a path should traverse (something that is not relevant in XPath as the only relation between the nodes of an XML tree is the parent-child relation which bears no explicit semantics).

Each step on the path, called a location step, follows the XPath location step syntax and is made of a) an optional axis declaration specifying the traversal direction in the directed graph, b) a type test taking the form of a URI reference represented as an XML qualified name (QName), or a \rdf{*} when the type is left unconstrained, c) optional predicates that specify further conditions on the nodes and arcs to be matched by this step.

The type test constrains property arcs to be labeled with the URI represented by the QName, or resource nodes to be instances of the class identified by this QName. In other words, type tests specify constraints on the types of properties and classes of resources to be traversed and selected by paths. Constraints on the URI of resources can be expressed as predicates associated with node location steps. A consequence of interpreting QName tests as type constraints is that FSL is syntactically and semantically compatible with Fresnel's basic selectors. The latter can therefore be considered a very limited subset of what can be expressed with FSL. Thus, any valid basic selector expression is a valid FSL expression.

More information about the language, including its grammar, data model and semantics can be found on the Fresnel Web site \cite{fsl05}. In the following examples, literals containing FSL expressions should all declare \rdf{fresnel:selector} as their datatype. This declaration has been omitted here for clarity.

\begin{small}
\begin{verbatim}
# A lens for foaf:Person resources that know at least five other resources
:PersonLens a fresnel:Lens ;
    fresnel:lensDomain "foaf:Person[count(foaf:knows) >= 5]".

# Show the foaf:name property of all foaf:Person
# instances known by the current resource.
:PersonLens a fresnel:Lens ;
    fresnel:showProperties ("foaf:knows/foaf:Person/foaf:name").
\end{verbatim}
\end{small}

\subsection{SPARQL}

The SPARQL RDF query language \cite{sparql05} offers the highest expressive power and can be used to specify lens and style domains. SPARQL queries used in this context must always return exactly one node set, meaning that only one variable is allowed in the query's SELECT clause. As with previous FSL examples, the literal's datatype declaration has been omitted.

\begin{small}
\begin{verbatim}
# A lens for John Doe's mailboxes
:PersonLens a fresnel:Lens ;
    fresnel:lensDomain "SELECT ?mbox WHERE ( ?x foaf:name 'John Doe' )
                                           ( ?x foaf:mbox ?mbox )".
\end{verbatim}
\end{small}




%--------------------------------------------------------------------
%--------------------------------------------------------------------
\section{Conclusion}

We have given an overview of Fresnel, a browser-independent, extensible vocabulary for modeling Semantic Web content presentation knowledge. Fresnel has been designed as a modularized, declarative language manipulating selection, formatting, and styling concepts that are applicable across representation paradigms, layout methods, and output formats. Fresnel core modules can be used to model presentation knowledge that is compatible and reusable between browsers and other types of Semantic Web information visualization~tools.

%I've added a ~ between the last two words to prevent tools from going on the next line.

Although core modules have been frozen for the time being, the Fresnel vocabulary remains a work in progress as new extension modules meeting special needs are being developed (e.g., for describing the {\em purpose} of lenses or adding new formatting capabilities). Extension modules are not necessarily aimed at being application- and paradigm-independent, as they might not be relevant in all cases; but their inclusion in Fresnel provides users with a unified framework for modeling presentation knowledge. 

Core modules are currently being implemented in various types of applications: SIMILE's Longwell \cite{simile} faceted browser, IsaViz \cite{isaviz} which represents RDF graphs as node-link diagrams, and Horus \cite{horus}, a PHP-based RDF browser. More information about Fresnel can be found on its web site\footnote{http://www.w3.org/2005/04/fresnel-info/}. Its development is an open, community-based effort and new contributors are welcome to participate in it.



%%%%%%%%%%%%%%%%%%%%%%%%%%%%%%%%%%%%%%%%%%%
% I'm temporarily using the \newpage and \vspace instructions to make sure
% that ACK and Refs fit on one page
%%%%%%%%%%%%%%%%%%%%%%%%%%%%%%%%%%%%%%%%%%%

\section*{Acknowledgments}

\vspace{-1mm}

\begin{small}
We would like to thank members of the SIMILE and Haystack projects at MIT for their valuable input, especially Stefano Mazzocchi, David Karger, Stephen Garland, David Huynh, Karun Bakshi, and the others who contributed to the discussion about Fresnel: Hannes Gassert, Jacco van Ossenbruggen, Dennis Quan, and Lloyd Rutledge.
\end{small}

\vspace{-3mm}

%
% ---- Bibliography ----
%

\bibliographystyle{splncs}
\bibliography{fresnel}

\end{document}
